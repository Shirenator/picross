\documentclass{report}


%%%%%%%%%%%%%%%%%%
%   Liste des packages utilisés  %
%%%%%%%%%%%%%%%%%%

% (oui y'en a 95% qui sont inutiles ^^)

\usepackage{amssymb}
\usepackage{array}
\usepackage{hyperref}
\usepackage{booktabs}
\usepackage{multirow}
\usepackage{float}
\usepackage{lmodern} %Pack de police
\usepackage{color}
\usepackage[dvipsnames]{xcolor}
\usepackage{graphicx}
\usepackage[utf8x]{inputenc}
\usepackage[T1]{fontenc}
\usepackage{natbib}
\usepackage[francais]{babel}
\usepackage{caption}
\usepackage{listings}
\usepackage{booktabs}
\usepackage[top=2cm, bottom=2cm,left=2cm, right=2cm]{geometry}
\usepackage{blindtext}
\usepackage{setspace}
\usepackage{graphicx}
\usepackage{titlesec, blindtext, color} % titres spéciaux + couleur pour les chapter

% on transforme les chapters en juste le numéro suivi du titre, avec un barre grisse
\definecolor{gray75}{gray}{0.75}
\newcommand{\hsp}{\hspace{20pt}}
\titleformat{\chapter}[hang]{\Huge\bfseries}{\thechapter\hsp\textcolor{gray75}{|}\hsp}{0pt}{\Huge\bfseries}

\begin{document}


%%%%%%%%%%%
%  Page de garde  %
%%%%%%%%%%%
\begin{titlepage}
	\begin{center}
	
		\begin{spacing}{1.5}
			Projet Picross\\
			\vspace*{\fill}
		\end{spacing}
		
		\begin{spacing}{2.5}
			\textbf{\Huge Application de création et d'aide à la résolution de puzzle \textit{picross}}\\[0.5cm]
			\textbf{\huge Cahier des charges} \\
			\vspace*{\fill}
			\textit{Étudiants :}
		\end{spacing}

		\begin{spacing}{1.15}
			\large
			\textsc{Brinon} Baptiste\\
			\textsc{Brocherieux} Thibault\\
			\textsc{Cohen} Mehdi\\
			\textsc{Debonne} Valentin\\
			\textsc{Lardy} Anthony\\
			\textsc{Mottier} Emeric\\
			\textsc{Pastouret} Gilles\\
			\textsc{Pelloin} Valentin\\
			\vspace*{\fill}
			\textbf{Groupe n°2} \\
			\textnormal{\large Licence Informatique\\ Le Mans Université\\ \today}
		\end{spacing}
		
	\end{center}
\end{titlepage}


%%%%%%%%%%
%    Sommaire    %
%%%%%%%%%%
\renewcommand{\contentsname}{Sommaire}
\tableofcontents


\chapter{Présentation}

	\section{Introduction}

		Dans le cadre de l'unité d'enseignant "Génie logiciel 2" de la Licence Informatique de Le Mans Université, les étudiants de troisième année sont amenés à travailler sur un projet de développement d'une application. 
	
  \section{Objectif de l'application}		
		Nous devons réaliser un jeu de type picross (aussi appelé nonogramme, logigramme ou hanjie) permettant à un utilisateur de résoudre des grilles et de l'aider dans sa réalisation.

	\section{Règles du picross}
		Le picross est un jeu de type puzzle. Le jeu consiste à découvrir un dessin sur une grille en noircissant des cases d'après des indices donnés.	
		Pour pouvoir déterminer les cases à colorier on dispose de groupe de nombres indiqués à chaque extrémité des lignes et des colonnes.
		\newline
		Les nombres indiqués permettent d'identifier la taille des blocs de cases à colorier sur la ligne ou colonne ainsi que leurs ordres.
		\newline
		Chaque groupe de cases indiqué doit être séparé des autres groupe de cases par une case blanche ou plus.
		
	\section{Une autre section ?}		
		
		Ce document décrit le contexte, les besoins fonctionnels, les objectifs et les contraintes définit par les clients. Un premier découpage des étapes nécéssaires à la réalisation du projet donne lieu dans le document à un planning prévisionnel.



\chapter{Spécification des besoins}

	\section{Mode de jeu}
			Le jeu sera composé de plusieurs chapitres de plus en plus difficiles. Chaque chapitre regroupera des grilles de différentes tailles. Au début du jeu, seul le premier chapitre sera débloqué et les autres se débloqueront au fur et à mesure que le joueur réussi des grilles.\\
			Un chapitre à part contiendra des niveaux différents des parties normales où la grille évoluera durant la partie. Dans ces niveaux, la taille de la grille augmentera au fur et à mesure que l'utilisateur complètera la grille existante.

		\section{Score}
			Le score d'un joueur sur une grille sera évalué par un nombre d'étoile. Un joueur pourra gagner quatre étoiles par grille au maximum. Le nombre d'étoile sera décerné au joueur lorsque celui-ci finit le niveau et sera calculé en fonction du temps de réalisation de cette même grille ainsi que du nombre d'aide utilisée. Chaque chapitre du jeu se débloquera automatiquement dès que le nombre d'étoile requis à son déverrouillage sera atteint. Plus un chapitre contiendra des grilles difficiles, plus le nombre d'étoile requis sera élevé.
				
		\section{Statistiques}
			En plus du score, l'application gardera en mémoire certaines statistiques pour le joueur ou pour une grille. Les statistiques du joueur seront son score total (nombre total d'étoile) et le nombre de niveau et chapitre réussis. Les statistiques pour chaque grille seront le meilleur temps effectué par le joueur, le nombre d'aides utilisées ainsi que le nombre d'erreurs effectuées lors de ce meilleur temps.
			
		\section{Aide}
			Plusieurs types d'aide seront proposés aux joueurs. Le type d'aide dépendra de la difficulté du chapitre. Pour un chapitre facile, l'aide coloriera directement une case correcte sur la grille. Pour un chapitre de difficulté moyenne, l'aide indiquera simplement la ligne ou la colonne où une case correcte qui n'est pas encore coloriée se trouve. Pour un chapitre difficile, l'aide indiquera seulement si le joueur à colorié une case incorrecte.\\
			Le joueur ne pourra pas utiliser autant d'aide qu'il le souhaite. Lorsque le joueur résoudra un picross, il obtiendra un nombre d'aide utilisable proportionnel au nombre d'étoiles qu'il a obtenues. En outre, plus le joueur avancera dans des chapitres difficiles, moins celui-ci sera authorisé à utiliser d'aides au cours d'un picross. De plus, l'utilisation d'une aide provoquera une pénalité de temps et donc moins d'étoiles. 

		\section{Hypothèses}			
			A tout moment durant une partie, le joueur pourra décider de partir dans le mode hypothèse. Durant ce mode, les cases qu'il remplira par la suite seront d'une autre couleur (la couleur de l'hypothèse). Il pourra créer jusqu'à cinq hypothèses, imbriquées les unes dans les autres.\\
			Si le joueur se rend compte que l'une de ses hypothèses est fausse, il pourra l'annuler, ce qui aura pour effet d'annuler toutes les autres hypothèses posées après celle annulée, et donc de revenir à l'état initial de l'hypothèse.\\
			En revanche, il pourra décider qu'une hypothèse est vraie. Dans ce cas, toutes les autres hypothèses qui interviennent avant celle-ci le deviennent aussi. Les cases placées changent alors de couleur, et deviennent des cases normales.
		
		\section{Didacticiel}
			Un niveau de l'application sera présent comme didacticiel. Ce niveau est optionnel et permettra aux joueurs débutants de se familiariser avec les règles du Picross. Il consistera en une partie guidée d'une grille simple de picross en expliquant au joueur les actions à réalisées étape par étape afin de finir la grille. L'application contiendra également un rappel des règles générales du picross.

		\section{Ergonomie}
			Durant une partie, le joueur pourra mettre le jeu en pause à n'importe quel moment. La grille sera masquée afin d'éviter toute tricherie et le chronomètre sera mis en pause également.\\
     			L'application pourra être utilisée intégralement à la souris et intégralement au clavier, au choix de l'utilisateur.\\
			Il sera également possible de sélectionner une suite de case verticale ou horizontale et de la coloriée d'un seul coup, à la souris comme au clavier. Lors cette sélection, le nombre de cases sélectionnées sera affiché.\\
			L'application sauvegardera automatiquement la partie après chaque action du joueur afin de pouvoir reprendre n'importe quelle grille à n'importe quel moment.

			
		\section{IHM}
			L'application sera disponible en français et en anglais et il sera possible de rajouter des langues supplémentaires.
			
chapter{Spécification des contraintes}

	\section{conception}
		Le client demande que le logiciel soit réalisé en ruby, et que la documentation soit engendrée via rdoc. Celle-ci sera livrée avec les livrables. L'interface doit être programmée en GtK.
		
	\section{cartes et interface}
		Une carte doit etre seulement en noir et blanc, Il serait sinon difficile de s'y retrouver avec les hypothèses qui sont elles aussi colorées.
		la grille doit prendre suffisemment de place sur la fenetre de manière à être facilement lisible malgrès l'importante quantitée de chiffres et l'interface, et ce, qu'importe la taille de la fenêtre. En lien avec cela, une taille de grille maximale de 25 par 25 à été mise en place.
		Les cartes doivent, une fois complétées, ressembler à quelquechose. Il doit s'agir d'images en noir et blanc ayant une signification.
	
	\section{limitation de l'aide}
		L'aide peut être utilisée de façon abusive. Pour éviter celà, une sanction est appliquée lors de l'utilisation de l'aide. Une perte de temps est appliquée lors de l'utilisation d'une aide, pouvant aller jusqu'a la suppression de toutes les étoiles remportées à la fin de la partie. La perte de temps sera indiquée separemment du temps que le joueur à mis à résoudre le picross, le total n'est pris en compte qu'au moment du calcul du score.
	
	\section{gameplay}
		lors d'une pause, le chrono est arrété. Le joueur pourrait ainsi résoudre le picross sans que le temps ne défile, ce qui briserai le système de scores. Pour éviter celà, un masquage des chiffres permettant la resolution du picross sera mis en place, de façon à ce que le joueur ne puisse résoudre le picross que durant que le chrono est actif.
		
	\section{Didacticiel}
		Bien qu'il soit jouable à n'importe quel moment, il ne doit pas être trop présent (pas de demarrage automatique au lancement du logiciel, uniquement un choix au menu) et doit uniquement contenir le necessaire pour expliquer toutes les règles ainsi que le fonctionnement du logiciel (explication de l'interface, système de score)
	
	\section{sauvegardes \& hypothèses}
		les sauvegardes sont réalisées uniquement à l'arret d'une partie, il n'est pas utile d'en réalisé d'autres. Les sauvegardes peuvent être utilisées pour le système de checkpoint. Il ne faut pas sauvegarder une hypothèse comme étant juste avant de l'avoir validée ( différenciation cases justes/cases de l'hypothèses). Les Checkpoints seront utilisables en quantité limitée car sinon il y aurait un désordre à l'affichage. Le maximum est fixé à 5 hypothèses simultanées, ce qui est suffisant.
		
	
	\section{délais}
		Le programme doit être terminé pour le 12 Avril 2018, ce qui nous laisse 3 mois pour le réaliser. L'utilisation de ce temps est indiqué sur le diagramme de Gantt suivant :
		
%		< insérer diagramme ici >
		
		
\chapter{livrables attendus}
	Les livrables rendus à l'issue de ce projet sont :
		-Le logiciel terminé et fonctionnel
		-Le cahier des charges
		-Le cahier de conception
		-Le manuel utilisateur
		-La documentation ( rdoc )
%	``I always thought something was fundamentally wrong with the universe'' \citep{adams1995hitchhiker}	
		\bibliographystyle{plain}
		\bibliography{references}

\end{document}
