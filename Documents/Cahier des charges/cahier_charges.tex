\documentclass{report}


%%%%%%%%%%%%%%%%%%
%   Liste des packages utilisés  %
%%%%%%%%%%%%%%%%%%

% (oui y'en a 95% qui sont inutiles ^^)

\usepackage{amssymb}
\usepackage{array}
\usepackage{hyperref}
\usepackage{booktabs}
 \usepackage{multirow}
\usepackage{float}
\usepackage{lmodern} %Pack de police
\usepackage{color}
\usepackage[dvipsnames]{xcolor}
\usepackage{graphicx}
\usepackage[utf8x]{inputenc}
\usepackage[T1]{fontenc}
\usepackage{natbib}
\usepackage[francais]{babel}
\usepackage{caption}
\usepackage{listings}
\usepackage{booktabs}
\usepackage[top=2cm, bottom=2cm,left=2cm, right=2cm]{geometry}
\usepackage{blindtext}
\usepackage{setspace}
\usepackage{graphicx}
\usepackage{titlesec, blindtext, color} % titres spéciaux + couleur pour les chapter

% on transforme les chapters en juste le numéro suivi du titre, avec un barre grisse
\definecolor{gray75}{gray}{0.75}
\newcommand{\hsp}{\hspace{20pt}}
\titleformat{\chapter}[hang]{\Huge\bfseries}{\thechapter\hsp\textcolor{gray75}{|}\hsp}{0pt}{\Huge\bfseries}

\begin{document}


%%%%%%%%%%%
%  Page de garde  %
%%%%%%%%%%%
\begin{titlepage}
	\begin{center}
	
		\begin{spacing}{1.5}
			Projet Picross\\
			\vspace*{\fill}
		\end{spacing}
		
		\begin{spacing}{2.5}
			\textbf{\Huge Application de création et d'aide à la résolution de puzzle \textit{picross}}\\[0.5cm]
			\textbf{\huge Cahier des charges} \\
			\vspace*{\fill}
			\textit{Étudiants :}
		\end{spacing}

		\begin{spacing}{1.15}
			\large
			\textsc{Brinon} Baptiste\\
			\textsc{Brocherieux} Thibault\\
			\textsc{Cohen} Mehdi\\
			\textsc{Debonne} Valentin\\
			\textsc{Lardy} Anthony\\
			\textsc{Mottier} Emeric\\
			\textsc{Pastouret} Gilles\\
			\textsc{Pelloin} Valentin\\
			\vspace*{\fill}
			\textbf{Groupe n°2} \\
			\textnormal{\large Licence Informatique\\ Le Mans Université\\ \today}
		\end{spacing}
		
	\end{center}
\end{titlepage}


%%%%%%%%%%
%    Sommaire    %
%%%%%%%%%%
\renewcommand{\contentsname}{Sommaire}
\tableofcontents


\chapter{Présentation}

	\section{Introduction}

		Dans le cadre de la Licence Informatique de Le Mans Université, les étudiants de troisième année sont amenés à élaborer un jeu de type picross (aussi appelé nonogramme, logigramme ou hanjie).
	
	\section{Règles du picross}
		Le picross est un jeu de type puzzle. Il est composé d'une grille. Soit les cases sont blanches (non-coloriées) soit noires (coloriées). 	
		Certaines de ces cases doivent-être coloriées afin de pouvoir révéler un dessin. Pour pouvoir déterminer les case à colorier on dispose de groupe de nombres indiqués à chaque extrémité des lignes et des colonnes.
		\newline
		Les nombres indiqués permettent d'identifier la taille des blocs de cases à colorier sur la ligne ou colonne ainsi que leurs ordres.
		\newline
		Chaque groupe de cases indiqué doit être séparé des autres groupe de cases par une case blanche ou plus.

	\section{Objectif de l'application}		
		
	\section{Une autre section ?}		
		
		Ce document a pour but



\chapter{Spécification des besoins}

	\section{Mode de jeu}
			Le jeu est composé de plusieurs chapitres. Chaque chapitre regroupe des grilles par taille. Au fur et à mesure que le joueur avance dans les chapitres, la difficulté augmente.
			\newline
			Puis dans chaque chapitre l'ordre d'apparition des grilles s'effectue en fonction de leur niveau de difficulté si celui-ci est existent.
			Il est proposé d'ajouter un mode de jeu "Progressif". Dans ce mode de jeu, la taille de la grille augmenterai au fur et à mesure que l'utilisateur complète la grille existante.

		
		\section{Score}
			Le score d'un joueur sur une grille est évalué par des étoiles. Un joueur peut gagner trois étoiles par grille au maximum. Le nombre d'étoiles qui seront décernées au joueur lorsque celui-ci finit le niveau est calculé en fonction du temps de réalisation de cette même grille ainsi que du nombre d'aide utilisé.
			
		\section{Statistiques}
			En plus du score, l'application gardera en mémoire certaines statistiques pour le joueur ou pour une grille. Les statistiques du joueur seront son score total (nombre total d'étoile) et le nombre de niveaux (et chapitres) réussis. Les statistiques pour chaque grille seront le meilleur temps effectué par le joueur, le nombre d'aides utilisées ainsi que le nombre d'erreur effectué lors de ce meilleur temps.
			
		\section{Aide}
			Plusieurs types d'aide seront proposés aux joueurs. Nous discernons trois type d'aide.
			\begin{itemize}
			    \item Une case a colorier peut être déterminer de façon certaine
			    \item Plusieurs groupe de cases se chevauchent, on peut déterminer un bloc qui sera colorié
			    \item Plusieurs combinaisons d'aide permettent de colorier une ou plusieurs cases.
			\end{itemize}
			Le joueur ne pourra pas utiliser autant d'aides qu'il le souhaite. Lorsque le joueur résout un picross, il obtient un nombre d'aides utilisables proportionnel au nombre d'étoiles qu'il a obtenu. En outre, plus le joueur avance dans des chapitres difficiles, moins celui-ci est autorisé à utiliser d'aides au cours d'un picross. + mode d'aide (à voir si c'est pas bcp de choses en même temps)

		\section{Hypothèses}			
			A tout moment, le joueur peut décider de partir dans une hypothèse. Les cases qu'il remplit par la suite sont d'une autre couleur (la couleur de l'hypothèse). Il doit pouvoir créer autant d'hypothèses qu'il veut, imbriquées les unes dans les autres.\\
			Si le joueur se rend compte que l'une de ses hypothèses est fausse, il peut l'annuler, ce qui aura pour effet d'annuler toutes les autres hypothèses posées après celle annulée, et donc de revenir à l'état initial de l'hypothèse.\\
			En revanche, il peut décider qu'une hypothèses est vraie. Dans ce cas, toutes les autres hypothèses qui interviennent avant celle-ci le deviennent aussi. Les cases placées changent alors de couleur, et deviennent des cases normales.
		
		\section{Didacticiel}
			+ le fait d'avoir les règles du jeu intégré dans l'application
		
		\section{Gameplay}
			le fait d'avoir un picross résolu qui doit représenter quelque chose (image, ...)
			possible pour le joueur de mettre en pause

		\section{Ergonomie}
			L'application doit pouvoir être utilisée intégralement à la souris, et intégralement au clavier, au choix de l'utilisateur.
			Il doit être possible de sélectionner une zone (verticale ou horizontale) et de la remplir d'un seul coup, à la souris comme au clavier.
			L'application sauvegarde automatiquement la partie après chaque action du joueur afin de pouvoir reprendre n'importe quelle grille à n'importe quel moment.
			
		\section{IHM}
			L'application sera disponible en français et en anglais.
			
chapter{Spécification des contraintes}

	\section{conception}
		Le client demande que le logiciel soit réalisé en ruby, et que la documentation soit engendrée via rdoc. Celle-ci sera livrée avec les livrables. L'interface doit être programmée en GtK.
		
	\section{cartes et interface}
		Une carte doit etre seulement en noir et blanc, Il serait sinon difficile de s'y retrouver avec les hypothèses colorées.
		Elle doit prendre suffisemment de place sur la fenetre de manière à être facilement lisible malgrès l'importante quantitée de chiffres, et ce qu'importe la taille de la fenêtre. En lien avec cela, une taille de grille maximale de 25 par 25 à été mise en place
		Les cartes doivent, une fois complétées, ressembler à quelquechose. Il doit s'agir d'images en noir et blanc ayant une signification.
	
	\section{limitation de l'aide}
		L'aide peut être utilisée de façon abusive. Pour éviter celà, une sanction est appliquée lors de l'utilisation de l'aide. Une perte de temps est appliquée lors de l'utilisation d'une aide, pouvant aller jusqu'a la suppression de toutes les étoiles remportées à la fin de la partie. La perte de temps sera indiquée separemment du temps que le joueur à mis à résoudre le picross, le total n'est pris en compte qu'au moment du calcul du score.
	
	\section{gameplay}
		lors d'une pause, le chrono est arrété. Le joueur pourrait ainsi le résoudre sans que le temps ne défile, ce qui briserai le système de scores. Pour éviter celà, un masquage des chiffres permettant la resolution du picross sera mis en place, de façon à ce que le joueur ne puisse résoudre le picross que durant que le chrono est actif.
		
	\section{Didacticiel}
		Bien qu'il soit jouable à n'importe quel moment, il ne doit pas être trop présent ( pas de demarrage automatique au demarrage, uniquement un choix au menu)
	
	\section{sauvegardes & hypothèses}
		les sauvegardes sont réalisées uniquement à l'arret d'une partie. 
		nombre de checkpoints limités à 5, chacun de couleur differente
			
\chapter{Conclusion}
%	``I always thought something was fundamentally wrong with the universe'' \citep{adams1995hitchhiker}	
		\bibliographystyle{plain}
		\bibliography{references}

\end{document}
